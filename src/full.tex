\documentclass{report}
\usepackage{setspace}
\doublespacing

\begin{document}

\tableofcontents{}

\chapter{Introduction}

\chapter{Phase-conjugate optical coherence tomography}

The past two decades have witnessed a number of experiments focused on the use of quantum-entangled states of light to achieve interesting effects. For example, quantum optical coherence tomography (Q-OCT) proposed by Abouraddy et al. and later demonstrated by Nasr et al., utilized maximally entangled photon states to achieve a 2X axial resolution improvement and dispersion cancellation over standard classical OCT.

Such effects were initially thought to be distinctly non-classical effects. However, much early research overlooked the theoretical possibility of classical light states that have no entanglement but are simply maximally correlated in the classical, stochastic sense. Recent advances in nonlinear optical crystals and a better understanding of parametric downconversion have permitted us to generate a variety of these interesting classical light fields that bear properties previously associated exclusively with quantum optics. Of particular note is the ability to generate classical biphoton states that are coincident in time with clasically maximal phase-sensitive correlations.

In this chapter we explore the use of these unusual classically-correlated states to implement phase-conjugate optical coherence tomography (PC-OCT) and demonstrate that we obtain many of the same advantages of Q-OCT previously attributed to quantum entanglement.

\section{Phase-sensitive light}

Phase-insensitive vs. phase-sensitive fields

\section{Spontaneous Parametric Downconversion basics}

Coupled mode equations

\subsection{Low-flux SPDC regime}

Pump/output dependence, entanglement

\subsection{High-flux SPDC regime}

Pump/output dependence in amplified regime, entanglement broken

\section{Experimental setup}

\subsection{Amplified SPDC source}

PM fiber laser connected to EDFA, power output curves, show we are operating in amplified regime

\subsection{Double-pass configuration}

Double pass achieved with waveplate tricks, mirror target, OPA, and matching the return with the pump in time

\subsection{Fiber dispersion and length}

Measurement of fiber length and dispersion, temperature dependence of index in fiber, thermal insulation

\subsection{Classical interferometer}

\subsection{Data collection}

\section{Results}

\section{Conclusions}

\chapter{Classical phase-sensitive ghost imaging}
We have explored the use of classical phase-sensitive light generated by SPDC in the high-flux regime, which allowed us to build a classical analog to quantum OCT which realizes many advantages previously thought to be exclusively quantum. Another experimentally feasible method of achieving classical phase-sensitive light fields is by imposing them with a pair of spatial light modulators driven with computer-generated, pseudorandom phase values.

Pittman et al. realized an imaging experiment, known later as ghost imaging, in which an entangled signal and idler photons were separated, with an bucket detector on one arm and a spatially-resolving detector on the other arm, and showing that it was possible to image an object placed in the arm with no spatial resolution by counting the coincidences between the two detectors. Although this mode of imaging was inspired by quantum entanglement, subsequent experiments showed that it was possible to achieve a similar imaging result using high-flux, pseudothermal classical light, a rotating ground-glass diffuser followed by a 50-50 beamsplitter, and observing the correlations in fluctuations between the two detectors.

Erkmen and Shapiro gave a unified theory of ghost imaging with Gaussian-state light that encompasses both biphoton and pseudothermal light sources. In the case of SPDC, a nonclassical phase-sensitive cross correlation between the signal and idler photons is exploited to achieve the ghost image; in the case of pseudothermal light sources, it is a phase-insensitive cross correlation between the two arms as imposed by the ground glass diffuser.

However, they also showed the possibility of using classical phase-sensitive light to achieve ghost imaging. In this chapter, we implement such a system by using spatial light modulators to impose phase-sensitive correlations between two arms.

\section{Ghost imaging theory}

\subsection{Ghost imaging with biphotons}

\subsection{Ghost imaging with pseudothermal light}

\subsection{Ghost imaging with phase-sensitive classical light}

\section{Spatial light modulators}

\subsection{Principle of operation}

\subsection{Pitfalls}

\section{Experimental setup}

\section{Results}

\section{Computational ghost imaging}

\section{Compressed ghost imaging}

\section{Computational compressed ghost imaging}

\section{Conclusions}

\chapter{Single-photon imaging}
We learned from experiments in ghost imaging that when realistic assumptions can be made about the target being imaged, such as few high spatial frequencies or few sharp edges, we are able to deduce the shape of the object much more quickly from a small number of measurement samples using compressed sensing rather than taking independent time averages of the cross-correlation at every pixel.

Building on this idea, we consider whether the same can be done for a basic 3-dimensional imaging experiment in which we collect a small sample of noisy data and computationally reconstruct the object by convex optimization rather than resorting to long data collection times, since the majority of practical imaging scenes are well-represented as a collection of piecewise smooth objects. In this chapter, we propose an imaging paradigm that uses only a single photon arrival per pixel to compute both depth and intensity images of an object in a fashion inspired by compressed sensing and convex optimization techniques. We then use our processing technique as a basis to propose more practical imaging device that uses a time-resolving single photon detector array.

\section{Sparsity in natural scenes}

\section{Computational reconstruction technique}

\section{First photon imaging}

\subsection{Experimental setup}

\subsection{Results}

\section{Fixed dwell time single photon imaging}

\subsection{Experimental setup}

\subsection{Results}

\section{Conclusions}

\chapter{Conclusions}

\end{document}
