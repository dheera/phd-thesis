\chapter{Introduction}
Over the latter half of the 20th century, much attention has been devoted to applying quantum-mechanical foundations previously laid by Heisenberg, Hilbert, Dirac, von Neumann, Schrodinger, and other early theoreticians. Many properties unique to quantum mechanics, including projective measurement, entanglement, no-cloning theorem, quantized eigenstates of systems, the uncertainty principle, and quantum descriptions of electromagnetic fields, led scientists to consider real-world applications of these quantum effects that are not realizable using classical physics. Today, many high-impact, widespread technologies are based on these quantum mechanical foundations, including magnetic resonance imaging, flash drive storage, laser technologies, and atomic clocks. Numerous other quantum techniuqes have been proposed and are highly active areas of research, including quantum cryptography, quantum computation, quantum error correction, and quantum imaging, with hopes that they will find widespread applications in the near future.

Most schemes for quantum sensing and measurement claim improvements over traditional classical imaging in a variety of properties ranging from resolution to privacy to background-free measurements. However, although it is often verifiable both theoretically and experimentally that the quantum versions of these experiments do indeed offer the proposed improvements, this is not a rigorous proof that they necessarily owe their improvements to quantum mechanical effects, and that there does not exist a classical but non-traditional experiment scheme to realize the same.

In this thesis we focus more specifically on quantum-mimetic imaging, the study of classical imaging experiments that demonstrate properties similar to their quantum counterparts. In particular, most biphoton sources, such as those using nonlinear crystals generate phase-sensitive coherence between the signal and idler beams. In contrast, most everyday classical light sources, including thermal light, sunlight, light-emitting diodes which are typically only capable of producing phase-insensitive coherence. Most quantum metrology experiments employ entangled biphoton sources, and therefore it is common to attribute their advantages to quantum entanglement. However, in many cases those advantages arise from phase-sensitive coherence alone, and in this work we will examine two such cases: quantum optical coherence tomography (OCT) and quantum ghost imaging.

In Chapters 2 and 3 we focus on these two experiments, respectively, using two different techniques to construct a classical phase-sensitive light source. Chapter 2 describes a phase-conjugate optical coherence tomography (OCT) system \cite{legouet-experimental} using an amplified downconversion-based classical phase-sensitive light source that realizes two of the key advantages afforded by quantum OCT \cite{lavoie-quantum} without utilizing quantum entanglement. Mazurek et al. \cite{mazurek-dispersion} later developed a scanning OCT imager based on classical phase-sensitive coherence that exploits similar principles. In Chapter 3 we develop a different classical phase-sensitive light source based on a pair of spatial light modulators \cite{venkatraman-classical} that mimics the properties of phase-sensitive ghost imaging, also previously thought to be uniquely quantum \cite{pittman-ghost}. We additionally demonstrate that with classical ghost imaging using spatial light modulators, it is possible to eliminate the reference arm and replace it with a computational simulation \cite{shapiro-discord}, showing that pseudothermal ghost images do not owe their images to quantum effects as was previously claimed \cite{ragy-nature}. In addition, we learn that classical ghost imaging has the additional advantage of being able to employ compressive sensing, a computational reconstruction techniqueearlier used in the Rice University single-pixel camera \cite{takhar-new,duarte-single} to make use of spatial correlations present in real-world scenes to increase acquisition speed by as much as a factor of 10. Zerom et al. \cite{zerom-entangled} later demonstrated a unique way to apply compressed sensing to biphoton ghost imaging. In this thesis, we successfully demonstrate both compressive ghost imaging and computational ghost imaging experiments. We then attempt to combine the two in a single experiment that makes use of both, finding that we are severely limited by inconsistencies in the fabrication of our SLMs, which also sheds light on the limitations of compressive sensing.

In Chapter 4 we describe a research collaboration with the Signal Transformation and Representation Group at MIT in which we take concepts learned from computational reconstruction in ghost imaging and apply them to a more general description of imaging. In particular, we jointly develop first-photon imaging, an imaging framework to acquire high-quality depth and reflectivity images of real-world scenes using only one photon per pixel of data \cite{kirmani-first}, where the main contribution of this thesis is in the experimental implementation and data acquisition. We then establish a second collaboration with the Politecnico di Milano, who developed a single-photon detector array \cite{villa-spad,scarcella-low} and experimentally implement a new version of the algorithm adapted for fixed-dwell time acquisition \cite{kirmani-photon}.

Finally, in Chapter 5 we conclude this thesis, summarize the novel contributions, and propose potential future extensions to the work presented here.
