\chapter{Introduction}
Over the latter half of the 20th century, much attention has been devoted to applying quantum-mechanical foundations previously laid by Heisenberg, Hilbert, Dirac, von Neumann, Schrodinger, and other early theoreticians. Many properties unique to quantum mechanics, including projective measurement, entanglement, no-cloning theorem, quantized eigenstates of systems, the uncertainty principle, and quantum descriptions of electromagnetic fields, led scientists to consider real-world applications of these quantum effects that are not realizable using classical physics. Today, many high-impact, widespread technologies are based on these quantum mechanical foundations, including magnetic resonance imaging, flash drive storage, laser technologies, and atomic clocks. Numerous other quantum techniuqes have been proposed and are highly active areas of research, including quantum cryptography, quantum computation, quantum error correction, and quantum imaging, with hopes that they will find widespread applications in the near future.

Most schemes for quantum sensing and measurement claim improvements over traditional classical imaging in a variety of properties ranging from resolution to privacy to background-free measurements. However, although it is often verifiable both theoretically and experimentally that the quantum versions of these experiments do indeed offer the proposed improvements, this is not a rigorous proof that they necessarily owe their improvements to quantum mechanical effects, and that there does not exist a classical but non-traditional experiment scheme to realize the same.

In this thesis we focus more specifically on quantum-mimetic imaging, the study of classical imaging experiments that demonstrate properties similar to their quantum counterparts. In particular, most biphoton sources, such as those using nonlinear crystals generate phase-sensitive coherence between the signal and idler beams. In contrast, most everyday classical light sources, including thermal light, sunlight, light-emitting diodes which are typically only capable of producing phase-insensitive coherence. Most quantum metrology experiments employ entangled biphoton sources, and therefore it is common to attribute their advantages to quantum entanglement. However, in many cases those advantages arise from phase-sensitive coherence alone, and in this work we will examine two such cases: quantum optical coherence tomography (OCT) and quantum ghost imaging.

In Chapters 2 and 3 we focus on these two experiments, respectively, using two different techniques to construct a classical phase-sensitive light source. Chapter 2 describes a phase-conjugate optical coherence tomography (OCT) system using an amplified downconversion-based classical phase-sensitive light source that realizes two of the key advantages afforded by quantum OCT without utilizing quantum entanglement. In Chapter 3 we develop a different classical phase-sensitive light source based on a pair of spatial light modulators that mimics the properties of phase-sensitive ghost imaging, also previously thought to be uniquely quantum. In addition, we learn that classical ghost imaging has the additional advantage of being able to employ compressed sensing, a computational reconstruction technique to make use of spatial correlations present in real-world scenes to increase acquisition speed by as much as a factor of 10.

In Chapter 4 we describe a research collaboration in which we take concepts learned from computational reconstruction in ghost imaging and apply them to a more general description of imaging. In particular, we develop first-photon imaging, an imaging framework to acquire high-quality depth and reflectivity images of real-world scenes using only one photon per pixel of data, and also begin development of a similar framework for acquisition of high-quality images using single-photon detector arrays.

Finally, in Chapter 5 we conclude this thesis, summarize the novel contributions, and propose potential future extensions to the work presented here.

TODO expand Kevin Resch, Howell et al
