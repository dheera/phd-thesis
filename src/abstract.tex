Many recent experiments have explored the use of nonclassical states 
of light to perform imaging or sensing. Although these experiments
require quantum descriptions of light to explain their behavior, the advantages they
claim are not necessarily unique to quantum light. This thesis
explores the underlying principles behind two of those imaging techniques and
realizes classical experiments that demonstrate properties similar to
their quantum counterparts.

The principal contributions of this thesis in the preceding quantum-mimetic imaging paradigm
are the experimental implementation of phase-conjugate optical
coherence tomography and phase-sensitive ghost imaging, two
experiments whose quantum counterparts utilize phase-sensitive light with nonclassical strength.
This thesis also explores the use of compressed sensing to further
speed up acquisition of ghost imaging.

Finally, a new paradigm inspired by compressed sensing is
demonstrated, in which high-quality depth and reflectivity images are
simultaneously captured using only the first photon arrival at each
pixel. This paradigm is also extended to the case of single-photon APD
arrays which may offer few-photon low-light imaging capabilities
beyond what is possible with current camera technologies.
