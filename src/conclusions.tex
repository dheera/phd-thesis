\chapter{Conclusions}

Quantum metrology, sensing, and imaging involve the use of nonclassical optical states or nonclassical detection schemes to obtain information about an object. Many of these configurations involve the use of entangled biphoton states in which joint measurements of the signal and idler beams cannot be described using semiclassical photodetection theory. Since the advent of quantum mechanics, a number of such configurations have been proposed and implemented, typically claiming measurement advantages over their classical counterparts, including resolution improvements, dispersion cancellation, and higher signal-to-noise ratios.

However, it is important to understand the fundamental principles that give rise to these advantages. In particular, although a particular experimental configuration may only be described by a strictly quantum description of light, that alone is not sufficient to justify that the claimed advantages are fundamentally quantum in nature, and that there does not exist a different, non-traditional classical experiment to achieve the same results. Such non-traditional experiments, which we refer to as quantum-mimetic, have been extensively analyzed over the past decade.

A primary motivation of this thesis has been to experimentally implement two of these quantum-mimetic configurations. In Chapter 2, we realized phase-conjugate OCT, which focuses on two advantages claimed of quantum OCT over classical OCT: a factor-of-2 axial resolution improvement and dispersion cancellation. As earlier theories showed that these improvements arise not from the quantum nature of entangled photons but simply the phase-sensitive cross-correlations between the signal and idler beams, we used SPDC in the amplified regime to construct a classical phase-sensitive light source that did not have any entanglement, and successfully realized these two advantages. Future experiments in PC-OCT may add a transverse scanning mechanism as well as image targets with features at multiple depths. Experimentally, the most difficult aspect of this extension would be to perform the phase conjugation, which requires both precise transverse alignment as well as a pulsed pump that is aligned with the back-reflected signal in time. In addition, multipath reflections within the target may necessitate additional signal processing steps before an OCT image can be recovered.

In Chapter 3, we turned to ghost imaging, another type of imaging initially thought to be formed exclusively due to the quantum nature of entangled photons. Although earlier theoretical and experimental works showed that ghost imaging is possible to implement using thermal light and classical detectors, no prior works have attempted to perform ghost imaging using phase-sensitive classical light. We implemented such a phase-sensitive source using a pair of spatial light modulators driven deterministically with pseudorandom phase patterns, creating a new type of phase-sensitive classical light source different from the SPDC-based source used for ghost imaging. We successfully demonstrated the inverted image signature characteristic of far-field ghost images produced by quantum phase-sensitive sources, and also demonstrated that using classical sources and detectors it was possible to operate at much higher fluxes, speeding up acquisition from several hours to mere minutes.

Imaging using spatial light modulators driven with deterministic phases also provided us with an opportunity to replace the entire reference arm with a computational simulation, which we successfully demonstrated. In addition, we realized that ghost imaging using a spatial light modulator is conceptually similar to imaging with a single-pixel camera, except that we use structured active illumination rather structured detection in the latter. Noting that our transmission masks were spatially sparse, having large continuous regions with few edges, we implemented compressive sensing methods on a simple beam-splitter based, phase-insensitive ghost imaging experiment, which afforded high-quality images while speeding up acquisition by up to a factor of 10 when compared with traditional averaging methods.

Compressive sensing makes use of the fact that most real-world objects are piecewise continuous, and exploits the spatial correlations between adjacent pixels to computationally reconstruct an image using a small number of acquisitions. In Chapter 4, we extended this concept beyond ghost imaging and considered its applications to mainstream imaging techniques. In particular, much attention recently has been focused on low-light reflectivity and depth imaging using Geiger-mode single photon counters, either by raster scanning the active illumination, or by employing detector arrays in a camera-like configuration. In either case, current imaging technology requires collecting tens to hundreds of photon arrivals at each pixel in order to obtain a high-quality image. However, similar to what we discovered with ghost imaging, most real-world scenes contain spatial structure and correlations between adjacent pixels, making it inefficient to treat pixels as independent processes. We established a research collaboration with the Signal Transformation and Representation Group at MIT and jointly developed ``first-photon imaging'', a novel low-light imaging paradigm in which we computationally reconstruct depth and reflectivity images using only the first photon arrival at each pixel, providing a factor-of-100 acquisition speedup over existing imagers using single-photon counters.

We also began the exploration of adapting our algorithm to the case of SPAD arrays, which is an active area of research and may eventually replace the need to raster-scan an image. Since in this case an entire image has a constant dwell time across all pixels, in contrast to the variable dwell time used for first-photon imaging, our reconstruction algorithm needed to be modif iedaccordingly. We presented preliminary results using a prototype fabricated by the Zappa group at the Politecnico di Milano. However, noting that the time resolution of these detectors was not a good fit for our laser pulse width and object feature size, future experiments may consider tweaking these parameters and imaging more suitable objects to better demonstrate the potential of our fixed-dwell time algorithm. Future SPAD arrays may also consider improvements in fill-factor; for example, Princeton Lightwave's current $128\times32$ and $32\times32$ Geiger-mode APD offerings \cite{princetonlightwave} feature integrated microlens arrays which increase the fill factor to 75\%. In addition, we hope that the future of SPAD array research brings higher-resolution arrays, both in number of pixels as well as time-bin resolution, which would hold immense possibilities for practical applications in biotechnology, medicine, mobile devices, space exploration, and military imaging, all of which have constraints on photon flux in either illumination or detection.

