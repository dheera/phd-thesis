First and foremost I am extremely grateful to my advisor Senior Research Scientist Dr. Franco Wong, without whom this research would not be possible, and to thesis committee members Prof.\ Jeffrey Shapiro and Prof.\ Vivek Goyal for being true mentors throughout the course of this work.

Dr. Wong was an extraordinary mentor who was highly inspirational and encouraging at every step along this research. He was not only extremely patient and available with every student, but also a quick and pragmatic thinker when any kind of difficulties arose in the laboratory. His guidance over the years was invaluable in the path to true step-by-step scientific research and in becoming a more independent thinker, communicator, and problem-solver.

Prof.\ Shapiro was an excellent teacher and his scientific rigor provided invaluable insight in understanding the theoretical foundations of our work. His presentations and comments at conferences, group meetings, and 6.453 lectures have always inspired me to strive harder to express concepts more clearly and concisely, and he always was willing to make time amidst his numerous responsibilities to jump in and provide his theoretical insights on any situation.

Embarking on cross-disciplinary projects has always been my true passion. Prof. Goyal, whose expertise is more concentrated on the signal processing side of our work, was highly supportive of our collaborative endeavor, and in the process was always patient in mentoring and exposing me to completely different viewpoints and approaches to scientific problems. The many cross-group discussions and debates were extremely insightful about not only our research project but also the possibilities that lie in the future for producing high-impact results through interdisciplinary collaborations.

I would like to thank all fellow group members, current and former, in no particular order: Zheshen Zhang, Sara Mouradian, Murphy Niu, Valentina Schettini, Maria Tengner, Ra\'{u}l Garc\'{i}a-Patr\'{o}n, Veronika Stelmakh, Tian Zhong, Taehyun Kim, Onur Kuzucu, Saikat Guha, Nicholas Hardy, Julian Le Gou\"{e}t, Baris Erkmen, and my research collaborators Ahmed Kirmani, Andrea Cola\c{c}o, Dongeek Shin, Rudi Lussana, for the numerous thought-provoking discussions in the office and laboratory, and the many insights that we have shared over the years which have helped me grow as a researcher and person.

Being able to express thoughts coherently and cultivate talent in younger students is also an indispensible part of humanity's scientific endeavors. I thank Prof. Dennis M. Freeman who offered me the opportunity to be a teaching assistant for 6.003, for his true enthusiasm and heart in exceptional teaching, and for setting an example of quality education. It was also through being a teaching assistant that I learned that although receiving an A in a course demonstrates ability to grasp subject matter from one angle, it is not until you try to teach the course to someone else that one truly and thoroughly understands the subject matter and the multiple valid ways of approaching each and every detail.

I would like to thank the numerous student and professional organizations on campus who have provided a support network, forged true friendships, and expanded my career horizons. These include the MIT-China Innovation and Entrepreneurship Forum, the MIT Sustainability Summit, the MIT Chinese Students and Scholars Association, the MIT Singapore Students Society, the MIT Republic of China (Taiwan) Students Association, and the Ashdown House Executive Committee, through all of which I have met numerous fellow MIT colleagues who have helped me develop the professional side of my career and fostered cross-disciplinary connections and opportunities to startups and industries worldwide. To the same extent I also thank all of my friends, too numerous to list here, who have supported me over the years.

Last but not least, I would also like to thank my parents and family for all their support both before and during my years at MIT.

This thesis marks the end of not only a doctoral program, but also a much longer chapter of my life at MIT including my earlier years as an undergraduate at the Institute. Although I spent much of my childhood moving from place to place, my decade at MIT has by far made the biggest impact on my life, both academically and personally. From late-night hacking on electronics projects to serious entrepreneurship discussions in the hallways, MIT has always been an open playground, both literally and figuratively, for both academic and extracurricular innovation, and it is by far the culture of openness, freedom, and mutual trust that I have found most helpful in fostering true synergies among the highly-motivated scientists, leaders, and innovators. I cannot express the gratitude I have for all the conveniences MIT provides, including effectively unlimited static IP addresses and no port restrictions, unlimited symmetric bandwidth, gigabit ethernet drops in every bedroom, open buildings and classrooms, laser cutters, machine shops, electronics supply rooms, piano practice rooms, and the myriad other resources that have been exceedingly useful both directly and indirectly to my doctoral research and further professional pursuits. The unique openness and willingness of MIT community members to trust, support, and mentor each other, form cross-disciplinary collaborations, share access to resources, and share expertise have also been invaluable to all of my developments at MIT.

This work was supported by the DARPA Quantum Sensors Program, the DARPA InPho program through the U.S. Army Research Office, and by the National Science Foundation.
