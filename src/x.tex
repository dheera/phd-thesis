\chapter{Phase-conjugate optical coherence tomography}

The past two decades have witnessed a number of experiments focused on the use of quantum-entangled states of light to achieve interesting effects. For example, quantum optical coherence tomography (Q-OCT) proposed by Abouraddy et al. and later demonstrated by Nasr et al., utilized maximally entangled photon states to achieve a 2X axial resolution improvement and dispersion cancellation over standard classical OCT.

Such effects were initially thought to be distinctly non-classical effects. However, much early research overlooked the theoretical possibility of classical light states that have no entanglement but are simply maximally correlated in the classical, stochastic sense. Recent advances in nonlinear optical crystals and a better understanding of parametric downconversion have permitted us to generate a variety of these interesting classical light fields that bear properties previously associated exclusively with quantum optics. Of particular note is the ability to generate classical biphoton states that are coincident in time with clasically maximal phase-sensitive correlations.

In this chapter we explore the use of these unusual classically-correlated states to implement phase-conjugate optical coherence tomography (PC-OCT) and demonstrate that we obtain many of the same advantages of Q-OCT previously attributed to quantum entanglement.

\section{Phase-sensitive light}

Phase-insensitive vs. phase-sensitive fields

\section{Spontaneous parametric downconversion}

Spontaneous parametric downconversion (SPDC) was first observed in the 1960's (1,2,3,4,5) and subsequently extensively studied (6,7,8,9,10) since then. More recently, SPDC has been of particular interest to quantum optics experiments, most notably used to generate heralded single photons and entangled photon sources (13, 14, 15, 16, 17). While SPDC is useful in the weakly-pumped regime for its quantum properties, few works have previously exploited the strongly-pumped regime for its strong classical phase coherence between the signal and idler beams which we will make use of in this work.

\section{Single-mode parametric fluorescence}

We would like to couple our SPDC outputs into a single-mode fibers for convenience. As SPDC output is typically highly multi-modal, a number of recent studies (12, 15, 23, 24, 25) have explored the idea of manipulating the focusing of the pump beam to concentrate the majority of SPDC output into a single spatial mode. Boyd and Kleinman found that the SPDC output power is maximized if $\Eta = L/b = 2.84$, where $L$ is the crystal length, $b = 2n \pi w_0^2 / \lambda$ is the confocal parameter in the crystal, $w_0$ is the beam waist, and $n$ is the crystal's index of refraction at vacuum wavelength $\lambda$. Note that the confocal parameter is identical for all three fields (pump, signal, idler).

There are three main factors which affect the spatial mode distribution of the SPDC output: (1) the natural divergence of the SPDC fields due to plane wave pumping over a finite crystal length, (2) angular divergence due to the focusing of the pump beam, and (3) the spectral bandwidth of the signal and idler measurement.

The first contribution can be derived from the longitudinal phase matching condition,
\begin{equation}
\Lambda k_z (\theta_{s,i}) = k_p - k_s \cos \theta_s - k_i \cos \theta_i - \frac{2\pi}{\Lambda} = k_s \theta_s ^2
\end{equation}
The phase-matching bandwidth is given by
\begin{equation}
\Delta k_z L = \pi
\end{equation}
where $L$ is the length of the crystal. We then obtain the natural divergence of the signal and idler fields:
\begin{equation}
\bar{\theta_{s,j}} = \sqrt{\frac{\lambda_s}{2 n_s L}}
\end{equation}

The second contribution is simply the divergence angle of the pump beam, i.e.
\begin{equation}
\theta_p = \frac{\lambda_p}{\pi n_p w_p}
\end{equation}
where $\lambda_p$ is the vacuum wavelength of the pump, $n_p$ is the index of the crystal at the pump wavelength, and $w_p$ is the beam waist of the pump at the focus.

The third contribution, which originates from the measurement spectral bandwidth of the signal and idler beams. is negligible if the measurement bandwidth is small compared to the phase-matching bandwidth, as will be the case in this experiment.

For single-mode generation of fluorescence, we try to equate the divergence angle due to pump focusing and the natural divergence angle ($\theta_p = \bar{\theta_s}$), i.e. the spot size of the pump at the waist is only large enough to support a single spatial mode of the degenerate signal and idler beams. This results in a pump focusing parameter of $\eta_p = \pi_2$. The total output divergence from both contributions is then given by $\theta = \sqrt{\bar{\theta_s}}^2$

\subsection{SPDC under strong pumping}

We see that in the low-flux regime, spontaneous parametric downconversion can be used to generate biphotons and entangled photon states. However, for the purpose of this experiment, we want to generate classically maximal phase-sensitive correlations without quantum entanglement. We show in this section that this is the case when the pump power is sufficiently high such that biphotons are amplified before exiting the crystal, breaking all entanglement properties.

\section{Experimental setup}

\subsection{Amplified SPDC source}

We use a home-built \cite{dheera_masters} 75-ps pulsed, polarization-maintaing Erbium-doped fiber laser at 1560 nm as a seed laser. The output is fed into an IPG Photonics Er-doped fiber amplifier with an average output power of ~5 W. We use a free-space optical setup \ref{figure_pcoct_0} to couple the output into a periodically-poled lithium niobate (PPLN) crystal to phase-matched for second harmonic generation (SHG) output at 780 nm with an average output power of ~3 W \ref{figure_pcoct_1}. We then use a second PPLN crystal in a Type I SPDC configuration to downconvert the 780 nm light into degenerate 1560-nm signal and idler outputs with anti-correlated phases.

\subsection{Double-pass configuration}

PC-OCT requires that the signal light passes through the same target twice before an interference measurement with the idler beam. We accomplish this double-pass configuration using a polarizing beam splitter (PBS) and waveplate as shown in figure \ref{figure_pcoct_2}.
...
After the second pass, the light is again vertically polarized and exits the PBS at 90 degrees.

In addition to the double-pass through the target, it is necessary to phase-conjugate the light. We accomplish using optical parametric amplification using a third PPLN crystal. However, it is necessary to ensure that the pulsed pump light and the pulsed return signal are matched in time in order to facilitate parametric amplification. As such we re-use the leftover pump signal from the second (SPDC) crystal but use a free-space variable delay to match the pulse in time with the return signal from the target, as seen in \ref{figure_pcoct_setup}.

\subsection{Fiber dispersion and length}

Measurement of fiber length and dispersion

\subsection{Temperature dependence of fiber index of refraction}

We observe that the index of refraction of fiber was significantly affected by changes in the ambient temperature; the fluctuations in room temperature by $\pm$1 degree were enough to cause several millimetres of difference in effective optical length after the length of fiber used in the experiment. We confirmed this by monitoring the changes in room temperature, as shown in graph \ref{figure_pcoct_3}, and alleviated the problem by placing the bulk of the fiber spool inside a thermally-insulating foam container.

\subsection{Classical interferometer}

Interferometric measurement was performed by combining both signal and reference arms using a 50/50 fiber beam splitter. A fiber circulator and free-space delay facilitate the ability to fine-tune the alignment between the two arms. A piezoelectric actuator was added to the free-space delay in order to repeatedly scan the interferometer over a full wavelength and measure the contrast at the two output arms.

\subsection{Data collection}

\section{Results}

Using a high reflectivity mirror as a target, figure \ref{figure_pcoct_result1} shows the results of three successive scans of the target with a relative separation of $\Delta z$ = 450 $\mu$m. Given the double-pass configuration, one would expect the resulting peaks to have a relative separation of $\Delta z_R = 2 \$Delta z$. giving us a 2 $\times$ axial improvement. The observed shifts were 920 $\pm$ 20 $\mu$m and 1660 $\pm$ 20 $\mu$m, confirming this behavior; the discrepancy in the right peak from the expected $\approx 1800$ $\mu$m is likely attributable to room temperature fluctuation as described earlier. Note that this effect is particularly noticable because we deliberately added long fibers to induce dispersion.

Besides the 2 $\times$ axial resolution improvement, we expect to observe cancellation of even-order dispersion in the signal arm. We used 77.1 metres and 71.2 metres of SMF-28 fiber before and after the target, respectively, leaving a difference of 5.9 meters in which dispersion was not cancelled.


\section{Conclusions}
