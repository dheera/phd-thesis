\chapter{Single-photon imaging}
We learned from experiments in ghost imaging that when realistic assumptions can be made about the target being imaged, such as few high spatial frequencies or few sharp edges, we are able to deduce the shape of the object much more quickly from a small number of measurement samples using compressed sensing rather than taking independent time averages of the cross-correlation at every pixel.

Building on this idea, we consider whether the same can be done for a basic 3-dimensional imaging experiment in which we collect a small sample of noisy data and computationally reconstruct the object by convex optimization rather than resorting to long data collection times, since the majority of practical imaging scenes are well-represented as a collection of piecewise smooth objects. In this chapter, we propose an imaging paradigm that uses only a single photon arrival per pixel to compute both depth and intensity images of an object in a fashion inspired by compressed sensing and convex optimization techniques. We then use our processing technique as a basis to propose more practical imaging device that uses a time-resolving single photon detector array.

\section{Sparsity in natural scenes}

\section{Computational reconstruction technique}

\section{First photon imaging}

\subsection{Experimental setup}

\subsection{Results}

\section{Fixed dwell time single photon imaging}

\subsection{Experimental setup}

\subsection{Results}

\section{Conclusions}
